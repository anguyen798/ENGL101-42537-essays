\documentclass[12pt]{article}
\usepackage{lipsum}
\usepackage[margin=1in]{geometry}
\usepackage{cite}
\usepackage{setspace}
\usepackage{times}
\usepackage{mla}

\doublespacing

\title{Blindsight and Echopraxia -- Dual Disabilities}
\author{Albert Nguyen}
\date{}

\begin{document}
\maketitle

Imagine you are aboard the space Theseus, its crew the bleeding edge of humanity augmented to give yourself in edge in dealing with Rorschach, a juvenile alien organic space vessel which appears to be unprepared and still growing. First contact with scramblers, Rorsarch's non-sentient octopoid inhabitants, creates the misinterpretation of an alien race of disabled cells-with-Waldoes \cite{blindsight} (Watts 226). Later their intelligence through processing speed far exceeds that of even the vampire captain Jukka Sarasti and coincidentally both share aspects of non-sentience \cite{blindsight} (Watts 309). You slowly realize that your consciousness, that voice in your head, is an evolutionary fluke, a parasite and a disability that causes your communication to be misinterpreted as a senseless waste of resources and therefore an act of war \cite{blindsight} (Watts 324). Rorsarch attacks by exploiting the disabilities of the Theseus crew. It continues its pattern of killing essential medical personnel with Issac Szpindel \cite{blindsight} (Watts 195), finishing off Robert Cunningham and disabling the Theseus crew but cutting off its escape \cite{blindsight} (Watts 330). Susan James multi-core augmentation is supplanted by a new panicking personality \cite{blindsight} (Watts 341), followed by Jukka Sarasti's pattern processing strength turning into a disability due to the crucifix glitch—an overload when right angles take up the majority of the vampire visual field and possibly Susan spiking his anti-Euclidian drugs that treat the seizures \cite{blindsight} (Watts 345). When all hope appears to be lost, it is Amanda Bates's disability of humanity in the act of noble self-sacrifice that allows the Theseus crew to defeat Rorsarch by mutually assured destruction, something the unconscious beings cannot comprehend due to following a set of rules to maximize their survival \cite{blindsight} (Watts 350). You have just imagined that you are Siri Keeton, cursed with comprehension that consciousness is an evolutionary disability and only to regain it on the return trip to Earth.

Throughout Blindsight (2006) and Echopraxia (2014), the central theme of the first is whether consciousness is a disability, while the focus of the second is transcending disability through augmentation. The anxieties of being obsolete are touched briefly in the first, phrased by Robert Cunnigham, "We can be utterly useless, or we can try and compete against the vampires and the constructs and the AIs" \cite{blindsight} (Watts 251). Theseus with its augmented team of post-humans \cite{blindsight} (Watts 49) operates at a disabled level when compared to vampires, who see in their heads prime numbers \cite{blindsight} (Watts 63), diffraction patterns \cite{blindsight} (Watts 203), quadrochromatia \cite{blindsight} (Watts 334), and draw strategic battle plans accounting for a multitude of variables to arrive at a specific timetable of thirty-seven minutes \cite{blindsight} (Watts 213). Human disability due to consciousness is further exemplified through the scramblers whose unconsciousness allows them to instantly on reflex calculate ten-digit prime numbers and complex shapes \cite{blindsight} (Watts 264), hide between literal eye saccades and induce agnosias "even their retarded children can rewire brains on the fly" \cite{blindsight} (Watts 283). In Echopraxia, this anxiety is compounded through the introduction of Bicammerals, hive-minded super monks with biological radios allowing telepathy \cite{echopraxia} (Echo 29), gaining these abilities through only through inflicting a disability on their bodies being the loss of their identity through undergoing induced brain cancer and synaptic pruning to form a new self \cite{echopraxia} (Echo 179).

\section*{Works Cited}
\begingroup
\renewcommand{\section}[2]{}%
\begin{thebibliography}{9}
\bibitem{blindsight}
Watts, Peter. \textit{Blindsight}. Tor Books, 2006.

\bibitem{echopraxia}
Watts, Peter. \textit{Echopraxia}. Tor Books, 2014.
\end{thebibliography}
\endgroup

\end{document}